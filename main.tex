\documentclass{article}
\usepackage{tikzsymbols} % For inserting CoffeeCup
\usepackage[hidelinks]{hyperref} % For making hyperlinks betwwen table of contents and sections; [hidelinks] - makes the link not standout
\usepackage{float} % Allow the use of [h]
\usepackage{amsmath} % For basic math operations
\usepackage{amssymb}

% For images and manipulation thereof
\usepackage{graphicx} % Basic image import and manipulation 
\graphicspath{Images/}
\usepackage{wrapfig} % For allowing images to be wrapped around text


% For modifying tables
\usepackage[table]{xcolor} % Allows the coloring of tables

% Control of the multi columns setup
\usepackage{multicol} % Used for making multicolumns
\setlength{\columnsep}{40pt} % Space between columns
\setlength{\columnwidth}{0.45\textwidth} % Width of the columns


% Setup for margens
\usepackage{geometry}
 \geometry{
 a4paper,
 total={170mm,257mm},
 left=20mm,
 top=20mm,
 }

 
% Custom commands
\newcommand{\bs}[1]{\textbackslash{}#1} % To use \bs{} for writing out commands
\newcommand{\myblock}{} % for making custom blocks 


% Creates the title for the document
\title{\LaTeX - Cheat Sheet \\
        \large A collection of useful latex codes for Computer Science \\
        Version: 1.0 \\
        Last updated: \today
}
\author{Joakim Iversen}
\date{}

\usepackage{titling}
\renewcommand\maketitlehooka{\null\mbox{}\vfill}
\renewcommand\maketitlehookd{\vfill\null}

\begin{document}

\begin{titlingpage}
    \maketitle
\end{titlingpage} % Making the title
\newpage

\tableofcontents % Inserting table of content
\newpage


\section{Introduction}
The creation of this document is for students studying Computer Science or interested in the depth of \LaTeX and the possibilities therein. \\
The document will have different section with commands and techniques to make the layout and styling of the paper easier and more pleasant to write. \\

If you have suggestions for things to add, or stumble across anything that will be useful to have in the sheet. Please contact: au761308@post.au.dk \\
\begin{center}
    Hope you enjoy using the Cheat Sheet \Coffeecup
\end{center}


\newpage

\section{Layout}
This section will have focus on some basic styling and layout in latex, such as styling text, inserting images and how to make different types of lists. \\


\subsection{Text}
Writing text in \LaTeX in the most basic form is just as easy as one might expect. You just have to write in the editor. 
But there are some basic functions we can add to this, to make things in the text stand more out. 

It is these function we will learn about in the following section. \\
The following section covers these usepackages and commands (Click on any part to get to the choosen subject): \\

\vspace{1.5em}
\noindent{
{\large\textbf{Usepackage mentioned in this section:}}}
\begin{table}[H]
    \noindent
    \centering
    \begin{tabular}{|c|c|} \hline
        \rowcolor{gray!30}
        Usepackage implementation & What the usepackage does \\ \hline
        \hyperref[subsubsec:text_family]{\bs{usepackage\{\textit{FONT-PACKAGE-NAME}\}}} & Used to import font for your document \\ \hline
    \end{tabular}
\end{table} % Table over covered usepackage for the text section


\subsubsection{Size}\label{subsubsec:text_size}
To set the font-size for your document you have to change the \bs{documentclass\{\ldots\}} to include the following \bs{documentclass[\textit{FONT-SIZE}]\{\ldots\}}. When declaring the font-size you have to use the unit "pt" which stands for points (A typography unit).\\


Besides setting the document font-size \LaTeX has 10 different size modifiers to use inside the basic text. These modifiers are the following: \\

\begin{table}[H]
    \centering
    \begin{tabular}{|c|c|}
    \hline
        \bs{tiny} & \tiny{Lorem Ipsum}  \\ \hline 
        \bs{scriptsize} & \scriptsize{Lorem Ipsum} \\ \hline
        \bs{footnotesize} & \footnotesize Lorem Ipsum \\ \hline
        \bs{small} & \small Lorem Ipsum\\ \hline
        \bs{normalsize} & \normalsize Lorem Ipsum\\ \hline
        \bs{large} & \large Lorem Ipsum \\ \hline
        \bs{Large} & \Large Lorem Ipsum \\ \hline
        \bs{LARGE} & \LARGE Lorem Ipsum \\ \hline
        \bs{huge} & \huge Lorem Ipsum \\ \hline
        \bs{Huge} & \Huge Lorem Ipsum  \\ \hline
    \end{tabular}
    \label{tab:font_size}
\end{table}
The sizing is gonna be active for the duration of the line you are typing. \footnotesize This text is therefor gonna be foot note size until i change it back using \bs{normalsize}. \normalsize

It is therefor not enough to make a line change by pressing enter or using the \textbackslash\textbackslash line break. \\

{\large\textbf{Examples:}} \\
\begin{multicols}{2}
\noindent  % No indentation for the first line
\vspace{0.5em}  % Add vertical space for better aesthetics
\textbf{Setup:} \\
\bs{documentclass[18pt]\{article\}}

\vspace{1.5em}  % Additional space below the box
\textbf{Output:} \\
\fbox{\begin{minipage}{\linewidth}
    {\fontsize{18}{15}\selectfont This command sets up a document with a font size of 12pt.}
\end{minipage}}
\end{multicols} % Example for \documentclass[FONT-SIZE]{article}

\begin{multicols}{2}
\noindent

\vspace{0.5em}
\textbf{Setup:} \\
\bs{tiny}

\vspace{0.5em}
\textbf{Output:} \\
\fbox{\begin{minipage}{\linewidth}
    \tiny This text is using the tiny command
\end{minipage}}
\end{multicols} % Example for \tiny command


 % Section about font size and how to change it throughout your document and set the font size from the start
\subsubsection{Family}\label{subsubsec:text_family}
The font-family is the type of font you use in your \LaTeX document. The standard font-family for a \LaTeX document is \textit{Computer Modern Typeface Family}. 
This includes different options for serif, sans serif and monospaced (Typewriter). You can switch between these throughout your document by using the following commands

\begin{table}[H]
    \centering
    \begin{tabular}{|c|c|}
        \hline
        \bs{textrm} & \textrm{Lorem Ipsum} \\ \hline
        \bs{textsf} & \textsf{Lorem Ipsum} \\ \hline
        \bs{texttt} & \texttt{Lorem Ipsum} \\ \hline
    \end{tabular}
    \label{tab:font_family}
\end{table} % Table for different basic font-family

If you want to see more fonts you can check under the section \hyperref[secc:font-family-sheet]{Miscellaneous and font-family}

When you have to use another font, as one shown under the section Miscellaneous, you have to import the "font package name" by implementing \bs{usepackage\{\textit{Font-Package-Name}\}}.
This allows you to use this font throughout your document.

If your otherwise just want to use a font in your text and not through out the whole document, you can use the code \bs{fontfamily\{\textit{Font-Code}\}}\bs{selectfont}
The first part of this code is to choose the font you want, where you have to insert the Font-Code as seen in the table in section \ref{tab:Big_Font_Family}.
The second part is then sets the font, this means that whitout the \bs{selectfont} the font will not be set. If you only want the font change for a specific part
of your text. you can do that the following way: \{\bs{usepackage\{\textit{Font-Package-Name}\}}\textit{The Text you want to be affected by the font}\} \\

{\large\textbf{Examples:}}
\begin{multicols}{2}
    \noindent
    \vspace{0.5em}
    \textbf{Setup:} \\
    \bs{usepackage}\{\textit{lmodern}\}
    \vspace{1em}

    \noindent
    \vspace{0.5em}
    \textbf{Output:} \\
    \fbox{\begin{minipage}{\linewidth}
        {\fontfamily{lmodern}\selectfont This Commands makes your document use the \textit{Latin Modern} font-family}
    \end{minipage}}
    \vspace{1em}

\end{multicols}

% Section about font family and how to change it in the document
\subsubsection{Styles}\label{subsubsec:text_styles}
There are some basic styles that we are used to having available to us in the text editors we are using on a normal basis like word and notepad.
The most basic of these are \textit{Italic}, \textbf{Bold}, and \underline{Underline}. 

To use these you have the following three commands

\begin{align*}
    \begin{tabular}{|c|c|} \hline
       \bs{textit} & \textit{This is Italic} \\ \hline
       \bs{textbf} & \textbf{This is Bold} \\ \hline
       \bs{underline} & \underline{This is underlined} \\ \hline
    \end{tabular}
\end{align*}
If you want to find more examples of way to style your text in \LaTeX you can check under Miscellaneous and \hyperref[secc:font-style-sheet]{font-style}

\subsection{Images}
The following section covers these usepackages and commands (Click on any part to get to the choosen subject): \\

\vspace{1.5em}
\noindent{
{\large\textbf{Usepackage mentioned in this section:}}}
\begin{table}[H]
    \noindent
    \centering
    \begin{tabular}{|c|c|} \hline
        \rowcolor{gray!30}
        Usepackage implementation & What the usepackage does \\ \hline
        \hyperref[tab:image_basic]{\bs{usepackage\{\textit{graphicx}\}}} & Used to add and manipulate images \\ \hline
        \hyperref[secc:image_wrapping]{\bs{usepackage\{wrapfig\}}} & Used to allow text wrapping around images \\ \hline 
    \end{tabular}
\end{table} % Table over covered usepackage for the text section

When working with images there are a few basic usepackages we want to have imported at the beginning of our document.

The following list covers the basic usepackages and commands: \\
\begin{table}[H] \label{tab:image_basic}
    \centering
    \begin{tabular}{|c|c|} \hline
        \bs{usepackage\{graphicx\}} & Allows us to insert and manipulate images in our document \\ \hline 
        \bs{graphicspath\{\textit{PATH-FOR-IMAGE-FOLDER}\}} & Informs of the folder where your images is stored \\ \hline
        \bs{includegraphics\{\textit{NAME-OF-IMAGE}\}} & Inserts an image into your document \\ \hline
    \end{tabular}
\end{table} % Table for the three most basic image commands.

With these three commands can we insert and modify images in our document. We use the \bs{includegraphics} to import the picture: \\
\includegraphics[height=8cm]{Images/Programming.png} \\
This is the most basic way of inserting an image. And there are a few different things we can do to make 
the image look better.

\subsubsection{Resizing and Rotating}
If we want to change the size or orientation of an image we can pass in different settings in the format as shown bellow

\begin{center}
\bs{includegraphics[\textit{FORMAT-SETTINGS}]\{ \ldots \} } \\
\end{center}

This means that you have to import the image into the document as normal but add the [\ldots] with your modifiers
before the image you're importing. 

There are "three" settings we can use with our image:
\begin{table}[H]
    \centering
    \begin{tabular}{|c|c|} \hline
        rotate=XXX & Rotates the image XXX degrees \\ \hline
        width=XXX & Sets the width of the image to XXX \\ \hline
        height=XXX & Sets the height of the image to XXX \\ \hline
        \rowcolor{gray!30}
        scale=XXX & Scales the image XXX-times \\ \hline
    \end{tabular}
\end{table} % the three format-settings

The gray box is not counted, since it has the same effect as the width, and height setting.
You can combine all these settings on the same image, by separating them with a ','. The values we give to the settings 
can be given with a selection of different units. To see the complete list you can look under Miscellaneous and \hyperref[secc:Image-Units]{Image Units}

\subsubsection{Positioning}
Now that we know how to insert images into our document using the \bs{includegraphics\{\}} command, we will
look into the positioning of the image. To do this we will have to use an environment called \textit{figure}.
(We will cover environments in a later section - so just ignore that for now)

\begin{multicols}{2}
    \noindent
    \begin{minipage}{0.70\linewidth}
        \noindent
        \vspace{0.5em}
        {\large\textbf{Setup:}} \\
        \textit{Example} \\
        \bs{begin\{figure\}[\textit{float-value}]} \\
        \bs{includegraphics[\textit{setting=value}]\{\textit{IMAGE-NAME}\}} \\
        \bs{end\{figure\}} \\[1em] % Add some vertical space for better readability

        \textit{Entered} \\
        \bs{begin\{figure\}[\textit{H}]} \\
        \bs{includegraphics[scale=.1]\{Images/Programming.png\}} \\
        \bs{end\{figure\}}
    \end{minipage}
    \columnbreak
   
        \noindent
        {\large\textbf{Output:}} \\
        \noindent
        \begin{figure}[H]
            \includegraphics[scale=0.05]{Images/Programming.png} % Adjust scale as needed
            \label{fig:programming}
        \end{figure}
\end{multicols} % Example of float and scale
Here we use the \textit{figure} environment to control the positioning of the image. The positioning is 
controlled by the \textit{float-value}. There are six different parameters that we can use to control the positioning:

\begin{table}[h]
    \centering
    \begin{tabular}{|c|p{5in}|} \hline
        \rowcolor{gray!30}
        Parameter & Position \\ \hline
        h & Place the float \textit{here} - approximately where it is placed in the source text (However, not exactly at the spot) \\ \hline
        t & Position at the \textit{top} of the page \\ \hline
        b & Position at the \textit{bottom} of the page \\ \hline
        p & Put on a special \textit{page} for float only \\ \hline
        ! & Override internal parameters \LaTeX uses for determining "good" float position \\ \hline
        H & Places the float at precisely the location in the \LaTeX source code. This requires the \textit{float} package,
        and though may cause problems occasionally. This is in a way the same as using h! \\ \hline
    \end{tabular}
\end{table} % Table of float parameters

Note that the '!' can be used in combination with the rest of the parameters, but not H.

\subsubsection{Wrapping of text} \label{secc:image_wrapping}
It's also possible to wrap text around a picture. To do this we will use the "wrapfigure" environment. 
To be able to use this you have to implement the usepackage "wrapfig". \\



\vspace{0.75em}
{\large\textbf{Setup:}} \\
\bs{begin\{wrapfigure\}[LINES]\{\textit{PLACEMENT-PARRAMETER}\}[OVERHANG]\{\textit{WIDTH}\}} \\
    \bs{centering} \\
    \bs{includegraphic[\textit{SETTING-PARAMETERS}]\{\textit{IMAGE-NAME}\}}
\bs{end\{wrapfigure\}} \\


\vspace{1.25em}

{\large\textbf{Output:}}


\begin{wrapfigure}[3]{l}{0.25\linewidth}
    \centering
    \includegraphics[width=0.9\linewidth]{Images/Programming.png}
    \label{fig:image_wrap}
\end{wrapfigure} % Wrapfig eksample with three narrow lines

This is a blank text to show to possibilities of wrapping a text around an image. This text has zero actual meaning, and is just trying to fill 
as much space as possible. 


\vspace{4\baselineskip}
Note that the wrapfig environment takes 4 different input parameters. The LINE parameter refers to the number of lines there should be narrowed besides the picture. 
In our example above we have set this to be 3, and as you can see, we have three lines that has been narrowed.

The next on is the placement of the image. Here we have 4 different values: \\

\begin{table}[h]
    \centering
    \begin{tabular}{|c|c|} \hline
        \rowcolor{gray!30}
        Position parameter & Placement of image \\ \hline
        r & Right side of the text \\ \hline
        l & Left side of the text \\ \hline 
        i & Inside edge-near the binding (in a \textit{twoside} document) \\ \hline
        o & Outside edge-far from the binding \\ \hline
    \end{tabular}
\end{table} % Table of the diffrent placements using a wrapfig
Each of these  does also have the equivalent letter in the uppercase version. When using the uppercase you allow the image to float, thus the lowercase version
means to place the image \textit{exactly} here.

The next parameter is the overhang. This is a fine-tuning of the image placement and can be used to give an image some "overhang" in the text area. 

The last parameter is the width of the width of the "box" given to the image inside the environment. Because this only refers to the box given to the image, and not the image itself.
It's good practice to let the image inside the wrapfig just be a tiny bit smaller than the wrapfig environment. That is how you make sure the text and image are not overlapping, and that you
get a small white space between image and text.

\subsubsection{Captioning, labelling and referencing}
When you have inserted a figure, image or anything else into our document we want to give the reader a small explanation to what
it is this figure resembles. To do this be use the code \bs{caption\{\textit{IMAGE-CAPTION}\}}.
We normally place the caption inside an environment with the figure we want to caption. 

\begin{multicols}{2}
    \vspace{0.5em}
    \noindent
    \textbf{Setup:} \\
    \bs{begin\{figure\}} \\
    \bs{includegraphic\{\textit{IMAGE-NAME}\}} \\
    \bs{caption\{Image of Computer\}} \\
    \bs{end\{figure\}} \\
    \columnbreak

    \vspace{0.5em}
    \noindent
    \textbf{Output:}
    \begin{figure}[H]
        \includegraphics[scale=0.1]{Images/Programming.png}
        \caption{Image of Computer}
        \label{fig:image_of_computer_caption}
    \end{figure}
\end{multicols}


As you can see we have the caption written out below the image along with \textit{Figure 1}.
The reason of the figure being numbered as number one, is because it is the first figure being captioned. \\

You also have a function in latex to label your figures and images. When doing this, you use the \bs{label\{\textit{CAPTION-TITLE}\}}
When we label our images and figures, it is not possible to see what you write in the label.
But the label allows you to reference a figure later in your document. You can choose your own system for labeling, but if you don't know what to do I have inserted
a table down below here, for how I in generally label my figures: \\

\begin{table}[h]
    \centering
    \begin{tabular}{|c|c|}
        \hline
        \rowcolor{gray!30}
        How I label & What figure \\ 
        \hline
        img:\textit{IMAGE\_DESCRIPTION} & Image \\ 
        \hline
        fig:\textit{FIGURE\_DESCRIPTION} & Figure \\ 
        \hline
        secc:\textit{SECTION\_DESCRIPTION} & Section \\ 
        \hline
        tab:\textit{TABLE\_DESCRIPTION} & Table \\ 
        \hline
        gra:\textit{GRAPH\_DESCRIPTION} & Graph \\ 
        \hline
    \end{tabular}
    \caption{Labeling Guide for Figures and Tables}
    \label{tab:caption_labels}
\end{table}


By using this system I know what it is my label is referring to, whether it is a figure, image, section or table. And this is useful 
when you want the reference a thing from your document. I always split the label up with '\_' if I use more words. This is just personal preference, and you can make a
label with spaces in it if you would like.

If you want to reference a figure you have a few different options to use, each having its own unique advantages

\begin{table}[h]
    \centering
    \begin{tabular}{|c|p{0.50\linewidth}|}
        \hline
        \rowcolor{gray!30}
        Command & Usage \\
        \hline
        \bs{ref\{\textit{FIG-NAME}\}} & Used to reference the figure number as shown below in the caption \\
        \hline
        \bs{pageref\{\textit{FIG-NAME}\}} & Used to reference the page number a figure is on \\
        \hline
        \bs{hyperref\{\textit{FIG-NAME}\}\{\textit{TEXT-TO-BE-SHOWN}\}} & Used to create a hyperlink between placement of the reference
         and the figure \\
        \hline
    \end{tabular}
\end{table}


\subsection{Lists}

\subsection{Environments}
\newpage

\section{Mathematics}
We will in this section look into different ways to do different mathematical methods such as symbols, operators, tables, graph and so on.

\subsection{Math Environment}
There are two groups of math Environment to use in a \LaTeX document. One for math inline with the text, and one for display.
But within each of these groups are three different ways of writing math from any of these two groups

\begin{table}[h]
    \centering
    \begin{tabular}{|c|c|} \hline
        \rowcolor{gray!30}
        Command & Group \\ \hline
        \$\textit{MATH}\$ & Inline \\ \hline
        \bs{(}\textit{MATH}\bs{)} & Inline \\ \hline
        \bs{begin\{math\}}\textit{MATH}\bs{end\{math\}} & Inline \\ \hline
        \bs{[\textit{MATH}}\bs{]} & Display \\ \hline
        \bs{begin\{displaymath\}} \textit{MATH} \bs{end\{displaymath\}} & Display \\ \hline
        \bs{begin\{equation\}} \textit{MATH} \bs{end\{equation\}} & Display \\ \hline
    \end{tabular}
    \label{tab:math_simple_operators}
\end{table}
This equation $E=MC^2$ is an example of an inline math equation.
\[E=MC^2\]
This is an example of a display equation.

These commands as stated in the table above will do precisely the same as long as they are in the same group. Therefor it is up to you which of these commands
you want to use, and what you think is going to make your \LaTeX code prettier and easier to read. \\

These enviroments is the foundation of writing math in \LaTeX, since it is not possible to write any math symbols, operators or anything outside these.

\subsection{Operators}
\LaTeX allows for all the different mathematic operators. The most basic can just be written by using the keyboard - but if youu want to enter some more text-based operators like $\log, \; \cos, \; \sin, \; \tan$, you have to use some commands. The basic operators, and how they will look can you see here.

\begin{table}[h]
   \centering
   \begin{tabular}{|c|c|} \hline
    \rowcolor{gray!30}
    Operator & Visual representation in Math environment \\ \hline
    + & $+$ \\ \hline
    - & $-$ \\ \hline
    / & $/$ \\ \hline
    * & $*$ \\ \hline
   \end{tabular} 
\end{table}

There are a few of these - for example the $*$ and $/$ - which has some prettier alternatives you can use. These are as follows

\begin{table}[h]
    \centering
    \begin{tabular}{|c|c|} \hline
        \rowcolor{gray!30}
        Alternative command & Result \\ \hline
        \bs{cdot} & $\cdot$ \\ \hline
        \bs{frac\{x\}\{y\}} & $\frac{x}{y}$ \\ \hline
    \end{tabular}
\end{table}

Using these commands does a small different, but will look much prettier in the long run. To get a longer list of all the different commands to the operators check under \hyperref[secc:Math-Operators]{Miscellaneous and Mathematics Operators} \\
\textbf{NB} That Many of the things in the table is just text based - But theres a difference between the plain text from writing it, and using the command shown in the table.
\subsection{Symbols}
One of the powers of \LaTeX is the ability to write many different symbols. This is especially useful when writing math, but can also be used in normal text. Most of the symbols can be written by using the normal name for the symbol for lower case and for uppercase write the first letter in uppercase. To see a larger list of symbols you can check under \hyperref[secc:Math-Symbols]{Miscellaneous and Mathematics symbols}. \\ 
\subsection{Tables}
\subsection{Graphs}
\newpage

% \section{Algorithm}
% \subsection{Pseudocode}
% \subsection{Tree}
% \newpage

% \section{References Library}
% \subsection{Setup}
% \subsection{Use}

% \section{Page Setup}
% \subsection{Front Page}
% \subsection{Table of Content}
% \subsection{Columns}
% \subsection{Header and Footer}
% \newpage

\section{Miscellaneous}
\subsection{Font}

    \subsubsection{Font-Family}
    \input{TexFiles/Font-Family Sheet.tex}
    \label{secc:font-family-sheet}

    \subsubsection{Font-Styles}
    \input{TexFiles/Font-Style Sheet.tex}
    \label{secc:font-style-sheet}
\subsection{Image}
\myblock{
    \subsubsection{Image Units}
    \input{TexFiles/Image-Units.tex}
    \label{secc:Image-Units}
}
    
\subsection{Mathematics}
    \subsubsection{Operators}
    \begin{table}[h]
    \centering
    \begin{tabular}{|c|c|} \hline
        \rowcolor{gray!30}
        Input & Output (In math Environment)\\ \hline
        + & $+$ \\ \hline
        - & $-$ \\ \hline
        * & $*$ \\ \hline
        \bs{cdot} & $\cdot$ \\ \hline
        / & $/$ \\ \hline
        \bs{frac\{x\}\{y\}} & $\frac{x}{y}$ \\ \hline
        \bs{sin} & $\sin$ \\ \hline
        \bs{cos} & $\cos$ \\ \hline
        \bs{tan} & $\tan$ \\ \hline
        \bs{log} & $\log$ \\ \hline
        \bs{min} & $\min$ \\ \hline
        \bs{max} & $\max$ \\ \hline
    \end{tabular}
\end{table}
    \label{secc:Math-Operators}

    \subsubsection{Symbols}
    \begin{table}[H]
    \centering
    \begin{tabular}{|c|c|c|c|} \hline
        \rowcolor{gray!30}
        Symbol & \LaTeX & Symbol & \LaTeX \\ \hline
        $A$ and $\alpha$ & A and \bs{alpha} & $N$ and $\nu$ & N and \bs{nu} \\ \hline
        $B$ and $\beta$ & B and \bs{beta} & $\Xi$ and $\xi$ & \bs{Xi} and \bs{xi} \\ \hline
        $\Gamma$ and $\gamma$ & \bs{Gamma} and \bs{gamma} & $O$ and $o$ & O and o \\ \hline 
        $\Delta$ and $/delta$ & \bs{Delta} and \bs{delta} & $\Pi$, $\pi$ and $\varpi$ & \bs{Pi}, \bs{pi} and \bs{varpi}\\ \hline
        $E$, $\epsilon$ and $\varepsilon$ & E, \bs{epsilon} and \bs{varpsilon} & $R$, $\rho$ and $\varrho$ & R, \bs{rho} and \bs{varrho} \\ \hline
        $Z$ and $\zeta$ & Z and \bs{zeta} & $\Sigma$, $\sigma$ and $\varsigma$ & \bs{Sigma}, \bs{sigma} and \bs{varsigma} \\ \hline
        $H$ and $\eta$ & H and \bs{eta} & $T$ and $\tau$ & T and \bs{tau} \\ \hline
        $\Theta$ and $\theta$ & \bs{Theta} and \bs{theta} & $\Upsilon$ and $\upsilon$ & \bs{Upsilon} and \bs{upsilon} \\ \hline
        $I$ and $\iota$ & I and \bs{iota} & $\Phi$ and $\phi$ & \bs{Phi} and \bs{phi} \\ \hline
        $K$ and $\kappa$ & K and \bs{kappa} & $C$ and $\chi$ & C and \bs{chi} \\ \hline
        $\Lambda$ and $\lambda$ & \bs{Lambda} and \bs{lambda} & $\Psi$ and $\psi$ & \bs{Psi} and \bs{psi} \\ \hline
        $M$ and $\mu$ & M and \bs{mu} & $\Omega$ and $\omega$ & \bs{Omega} and \bs{omega} \\ \hline
    \end{tabular}
    \caption{Greek Letters}
\end{table}

\begin{table}[H]
    \centering
    \begin{tabular}{|c|c|c|c|c|c|} \hline
        \rowcolor{gray!30}
        Symbol & \LaTeX & Comment & Symbol & \LaTeX & Comment \\ \hline
        $<$ & \textless & is less than & $>$ & \textgreater & is greater than \\ \hline
        $\nless$ & \bs{nless} & is not less than & $\ngtr$ & \bs{ngtr} & is not greater than \\ \hline
        $\leq$ & \bs{leq} & is less than or equal to & $\geq$ & \bs{geq} & is greater than or equal to \\ \hline
        $\leqslant$ & \bs{leqslant} & is less than or equal to & $\geqslant$ & \bs{geqslant} & is greater than or equal to \\ \hline
        $\nleq$ & \bs{nleq} & is not less than or equal to & $\ngeq$ & \bs{ngeq} & is not greater than or equal to \\ \hline
        $\nleqslant$ & \bs{nleqslant} & is not less than or equal to & $\ngeqslant$ & \bs{ngeqslant} & is not greater than or equal to \\ \hline
        $\prec$ & \bs{prec} & precedes & $\succ$ & \bs{succ} & succeds \\ \hline
        $\nprec$ & \bs{nprec} & does not precede & $\nsucc$ & \bs{nsucc} & does not succeed \\ \hline
        $\preceq$ & \bs{preceq} & precedes or equals & $\succeq$ & \bs{succeq} & succeds or equals \\ \hline
        $\npreceq$ & \bs{npreceq} & neither precedes nor equal & $\nsucceq$ & \bs{nsucceq} & neither succeeds nor equal \\ \hline
        $\ll$ & \bs{ll} & & $\gg$ & \bs{gg} & \\ \hline
        $\lll$ & \bs{lll} & & $\ggg$ & \bs{ggg} & \\ \hline
        $\subset$ & \bs{subset} & is a proper subset of & $\supset$ & \bs{supset} & is a proper superset of \\ \hline
        $\not\subset$ & \bs{not\bs{subset}} & is ot a proper subset of & $\not\supset$ & \bs{not\bs{supset}} & is not a proper super set of \\ \hline
        $\subseteq$ & \bs{subseteq} & is a subset of & $\supseteq$ & \bs{supseteq} & is a superset of \\ \hline
        $\nsubseteq$ & \bs{nsubseteq} & is not a subset of & $\nsupseteq$ & \bs{nsupseteq} & is not a superset of \\ \hline
        $\sqsubset$ & \bs{sqsubset} & & $\sqsupset$ & \bs{sqsupset} & \\ \hline
        $\sqsubseteq$ & \bs{sqsubseteq} & & $\sqsupseteq$ & \bs{sqsupseteq} & \\ \hline
        \rowcolor{gray!30}
        & & & & & \\ \hline
        $=$ & = & is equal to & $\doteq$ & \bs{doteq} & \\ \hline
        $\equiv$ & \bs{equiv} & is equivalent to & $\approx$ & \bs{approx} & is approximately \\ \hline
        $\cong$ & \bs{cong} & is congruent to & $\simeq$ & \bs{simeq} & is similar or equal to \\ \hline
        $\sim$ & \bs{sim} & is similar to & $\propto$ & \bs{propto} & is proportional to \\ \hline
        $\neq$ or $\ne$ & \bs{neq} or \bs{ne} & is not equal to & & & \\ \hline
        \rowcolor{gray!30}
        & & & & & \\ \hline
        $\parallel$ & \bs{parallel} & is parallel with & $\nparallel$ & \bs{nparallel} & is not parallel with \\ \hline
        $\asymp$ & \bs{asymp} & is asymptotic to & $\bowtie$ & \bs{bowtie} & \\ \hline
        $\vdash$ & \bs{vdash} & & $\dashv$ & \bs{dashv} & \\ \hline
        $\in$ & \bs{in} & is member of & $\ni$ & \bs{ni} & owns, has member \\ \hline
        $\smile$ & \bs{smile} & & $\frown$ & \bs{frown} & \\ \hline
        $\models$ & /bs{models} & models & $\notin$ & \bs{notin} & is not member of \\ \hline
        $/perp$ & \bs{perp} & is perpendicular with & $\mid$ & \bs{mid} & divides \\ \hline
    \end{tabular}
    \caption{Relation Operators} 
\end{table}

\begin{table}[H]
    \centering
    \begin{tabular}{|c|c|c|} \hline
        \rowcolor{gray!30}
        Symbol & \LaTeX & Comment \\ \hline
        + & $+$ & \\ \hline
        - & $-$ & negation \\ \hline
        $neg$ & \bs{neg} & not \\ \hline
        ! & $!$ & factorial \\ \hline
        $\#$ & \bs{\#} & primorial \\ \hline
    \end{tabular}
    \caption{Unary Operators}
\end{table}

\begin{table}[H]
    \centering
    \begin{tabular}{|c|c|c|c|c|c|} \hline
        \rowcolor{gray!30}
        Symbol & \LaTeX & Comment & Symbol & \LaTeX & Comment \\ \hline
        $\pm$ & \bs{pm} & plus or minus & $\cap$ & \bs{cap} & Set intersection \\ \hline
        $\diamond$ & \bs{diamond} & & $\oplus$ & \bs{oplus} & \\ \hline
        $\mp$ & \bs{mp} & Minus or plus & $\cup$ & \bs{cup} & Set union \\ \hline
        $\bigtriangleup$ & \bs{bigtriangleup} & & $\ominus$ & \bs{ominus} & \\ \hline
        $\times$ & \bs{times} & Multiplied by & $\uplus$ & \bs{uplus} & Multiset addition \\ \hline
        $\bigtriangledown$ & \bs{bigtriangledown} & & $\otimes$ & \bs{otimes} & \\ \hline
        $\div$ & \bs{div} & Divided by & $\sqcap$ & \bs{sqcap} & \\ \hline
        $\triangleleft$ & \bs{triangleleft} & & $\oslash$ & \bs{oslash} & \\ \hline
        $\ast$ & \bs{ast} & Asterisk & $\sqcup$ & \bs{sqcup} & \\ \hline
        $\triangleright$ & \bs{triangleright} & & $\odot$ & \bs{odot} & \\ \hline
        $\star$ & \bs{star} & & $\vee$ & \bs{vee} & \\ \hline
        $\bigcirc$ & \bs{bigcirc} & & $\circ$ & \bs{circ} & \\ \hline
        $\dagger$ & \bs{dagger} & & $\wedge$ & \bs{wedge} & \\ \hline
        $\bullet$ & \bs{bullet} & & $\setminus$ & \bs{setminus} & Set difference \\ \hline
        $\ddagger$ & \bs{ddagger} & & $\cdot$ & \bs{cot} & \\ \hline
        $\wr$ & \bs{wr} & & $\amalg$ & \bs{amalg} & \\ \hline
    \end{tabular}
    \caption{Binary Operators}
\end{table}
    \label{secc:Math-Symbols}

% \subsection{Emojies}
% \subsection{Custom Commands}
\newpage

% \section{Error Messages}

\end{document}
